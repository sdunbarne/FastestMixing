%%% -*-LaTeX-*-
%%% fastestmixing.tex.orig
%%% Prettyprinted by texpretty lex version 0.02 [21-May-2001]
%%% on Fri Oct 28 09:58:58 2022
%%% for Steve Dunbar (sdunbar@family-desktop)

\documentclass[12pt]{article}

\input{../../../../etc/macros}
%% \input{../../../../etc/mzlatex_macros}
\input{../../../../etc/pdf_macros}

\bibliographystyle{plain}

\begin{document}

\myheader \mytitle

\hr

\sectiontitle{Fastest Mixing Markov Chain}

\hr

\usefirefox

% \hr

% \visual{Study Tip}{../../../../CommonInformation/Lessons/studytip.png}
% \section*{Study Tip}

\visual{Rating}{../../../../CommonInformation/Lessons/rating.png}
\section*{Rating} %one of
% Everyone: contains no mathematics.
% Student: contains scenes of mild algebra or calculus that may require guidance.
% Mathematically Mature: may contain mathematics beyond calculus with proofs.
Mathematicians Only:  prolonged scenes of intense rigor.

\hr

\visual{Section Starter Question}{../../../../CommonInformation/Lessons/question_mark.png}
\section*{Section Starter Question}

If an irreducible aperiodic Markov chain is symmetric, that is,
represented by a symmetric matrix, then what is the specific stationary
distribution?  What determines the rate at which the Markov chain
approaches the stationary distribution?

\hr

\visual{Key Concepts}{../../../../CommonInformation/Lessons/keyconcepts.png}
\section*{Key Concepts}

\begin{enumerate}
    \item
        The asymptotic rate of convergence of the Markov chain to the
        stationary distribution depends on the second-largest eigenvalue
        modulus, called the mixing rate.  The mixing rate is denoted by \(
        \mu \) or \( \mu(P) \) where \( P \) is the transition
        probability matrix.
    \item
        If \( P \) is an \( n \times n \) symmetric stochastic matrix,
        then
        \[
            \mu(P) = \| P - (1/n) \vect{1}\vect{1}^T \|_2
        \] where \( \| \cdot \|_2 \) denotes the spectral norm.
    \item
        The eigenvalues and eigenvectors of the tridiagonal matrix
        \[
            P_{1/2} =
            \begin{pmatrix}
                1/2 & 1/2 & 0 & & \\
                1/2 & 0 & 1/2 & & \\
                & \ddots & \ddots & \ddots& \\
                & & 1/2 & 0 & 1/2 \\
                & & & 1/2 & 1/2
            \end{pmatrix}
        \] can be determined by solving a recursive system of equations
        and are \( \lambda_j = \cos\left( \frac{(j-1) \pi}{n} \right) \)
        for \( j = 1,\dots, n \).  In particular, the largest eigenvalue
        is \( 1 \).
    \item
        The mixing rate \( \mu(P_{1/2}) = \cos(\pi/n) \) for \( P_{1/2} \)
        is the smallest among all symmetric stochastic tridiagonal
        matrices.
\end{enumerate}

\hr

\visual{Vocabulary}{../../../../CommonInformation/Lessons/vocabulary.png}
\section*{Vocabulary}
\begin{enumerate}
    \item
        A matrix for which the row sums \( \sum_{j} P_{ij} = 1 \) is a
        \defn{stochastic matrix}.
    \item
        The asymptotic rate of convergence of the Markov chain to the
        stationary distribution depends on the second-largest eigenvalue
        modulus of \( P \), called the \defn{mixing rate}.
    \item
        The \defn{spectral norm}, also called the operator norm, is the
        natural norm of a matrix induced by the \( L^2 \) or Euclidean
        vector norm.  The spectral norm is also the maximum singular
        value of the matrix, that is the square root of the maximum
        eigenvalue of \( A^H A \) where \( A^H \) is the conjugate
        transpose.
\end{enumerate}

\hr

\section*{Notation}
\begin{enumerate}
    \item
        \( n \) -- number of nodes in the network
    \item
        \( i, j \) -- typical nodes in the network
    \item
        \( P_{ij} \) -- transition probability from \( i \) to \( j \)
    \item
        \( P \) -- transition probability matrix
    \item
        \( P_{1/2} \) -- tridiagonal transition probability matrix with
        every entry \( 1/2 \).
    \item
        \( \vect{1} \) -- the \( n \times 1 \) vector whose entries are
        all \( 1 \)
    \item
        \( 1 = \lambda_1 \ge \lambda_2 \ge \cdots \ge \lambda_n \ge -1 \)
        -- the eigenvalues ordered by nonincreasing magnitude
    \item
        \( \mu(P) = \max_{i=2,\dots,n} | \lambda_i(P) | = \max(\lambda_2,
        -\lambda_n) \) -- the mixing rate
    \item
        \( A^H \) -- conjugate transpose of matrix \( A \)
    \item
        \( \vect{u}^{(2)}, \dots \vect{u}^{(n)} \) -- the \( n-1 \)
        eigenvectors of \( P \) corresponding to the eigenvalues \(
        \lambda_2, \dots \lambda_n \)
    \item
        \( \vect{y}, \vect{z} \) -- vectors in \( \mathbb{R}^n \)
    \item
        \( s_{-}, s_{+} \) -- solutions of the quadratic characteristic
        function.
\end{enumerate}

\visual{Mathematical Ideas}{../../../../CommonInformation/Lessons/mathematicalideas.png}
\section*{Mathematical Ideas}

This section considers the problem of assigning transition probabilities
to the edges of a simple linear network in such a way that the resulting
Markov chain mixes as rapidly as possible.  The problem is equivalent to
assigning probabilities to a random walk with partially reflecting
boundaries with each step either to the vertex itself or its nearest
neighbor.  It is also equivalent to certain urn models.  The article
proves the fastest mixing is obtained when each edge has a transition
probability of \( 1/2 \).  This result is intuitive.

\subsection*{Introduction, Motivation, Application}

Consider a network (or mathematical graph) with \( n \ge 2 \) vertices,
labeled \( 1,2, \dots n \) with \( n-1 \) edges connecting adjacent
vertices and with a loop at each vertex, as in Figure~%
\ref{fastestmixing:fig:mixing_graph}.  Consider the Markov chain, or
random walk, on this graph%
\index{random walk}%
, with transition probability from vertex \( i \) to vertex \( j \)
denoted \( P_{ij} \).  The requirement that transitions can occur only
on an edge or loop of the graph is equivalent to \( P_{ij} = 0 \) when \(
|i -j| > 1 \).  Thus \( P \) is a tridiagonal matrix.%
\index{tridiagonal matrix}
Since \( P_{ij} \) are transition probabilities, \( P_{ij} \ge 0 \) and \(
\sum_{j} P_{ij} = 1 \).  Since the row sums are \( 1 \), \( P \) is a
\defn{stochastic matrix}.%
\index{stochastic matrix}
In matrix-vector terms,
\begin{equation}
    P \vect{1} = \vect{1}%
    \label{fastestmixing:eq:eigenvalue_one}
\end{equation}
where \( \vect{1} \) is the \( n \times 1 \) vector whose entries are
all \( 1 \).  Therefore, \( 1 \) is an eigenvalue of the matrix \( P \).

\begin{figure}
    \centering
\begin{asy}
settings.outformat = "pdf";

size(5inches);

real myfontsize = 12;
real mylineskip = 1.2*myfontsize;
pen mypen = fontsize(myfontsize, mylineskip);
defaultpen(mypen);

real marge= 1mm;
pair z1 = (1,0), z2 = (3,0), z3 = (5,0);
pair znm2 = (8,0), znm1 = (11,0), zn = (14,0);
transform r=scale(1.0);

object state1 = draw("$1$", roundbox, z1, marge),
       state2 = draw("$2$", roundbox, z2, marge),
       state3 = draw("$3$", roundbox, z3, marge),
statenm2 = draw("$\scriptstyle{n-2}$", roundbox, znm2, marge),
       statenm1 = draw("$\scriptstyle{n-1}$", roundbox, znm1, marge),
       staten = draw("$n$", roundbox, zn, marge);

add(new void(picture pic, transform t) {
    draw(pic, r*Label("$P_{12}$"), point(state1, E, t)..point(state2, W, t), Arrows);
    draw(pic, r*Label("$P_{23}$"), point(state2, E, t)..point(state3, W, t), Arrows);

    draw(pic, r*Label(""), point(state3, E, t)..point(statenm2, W, t), dashed, Arrows);

    draw(pic, Label("$P_{n-2, n-1}$"), point(statenm2, E, t)..point(statenm1, W, t), Arrows);
    draw(pic, Label("$P_{n-1, n}$"), point(statenm1, E, t)..point(staten, W, t), Arrows);
 });

add(new void(picture pic, transform t) {
    draw(pic, r*Label("$P_{11}$"), point(state1, NE, t){NE}..point(state1, NW, t), Arrow);
    draw(pic, r*Label("$P_{22}$"), point(state2, NE, t){NE}..point(state2, NW, t), Arrow);
    draw(pic, r*Label("$P_{33}$"), point(state3, NE, t){NE}..point(state3, NW, t), Arrow);

    draw(pic, Label("$P_{n-2, n-2}$"), point(statenm2, NE, t){NE}..point(statenm2, NW, t), Arrow);
    draw(pic, Label("$P_{n-1, n-1}$"), point(statenm1, NE, t){NE}..point(statenm1, NW, t), Arrow);
    draw(pic, Label("$P_{n,   n  }$"), point(staten,   NE, t){NE}..point(staten,   NW, t), Arrow);
 });

label("$P_{21}$", (z1+z2)/2, 1.5N);
label("$P_{32}$", (z3+z2)/2, 1.5N);

label("$P_{n-1,n-2}$", (znm2+znm1)/2, 1.5N);
label("$P_{n,n-1}$", (zn+znm1)/2, 1.5N+0.1E);
\end{asy}
    \caption{A random walk graph with transition probabilities.%
    \label{fastestmixing:fig:mixing_graph}}
\end{figure}

Furthermore, assuming the transition probabilities are symmetric, so
that \( P_{ij} = P_{ji} \) means \( P \) is a symmetric,
doubly-stochastic, tridiagonal matrix.  Since \( P \vect {1} = \vect{1} \),
then \( (1/n) \vect{1}^T P = (1/n) \vect{1}^T \), and the uniform
distribution is a stationary distribution for the Markov chain.

The asymptotic rate of convergence of the Markov chain to the stationary
distribution depends on the second-largest eigenvalue modulus of \( P \),
called the \defn{mixing rate}.%
\index{mixing rate}
Since \( P \) is symmetric, the eigenvalues of \( P \) are real, and
bounded in magnitude by \( 1 \).  Ordering the eigenvalues by
nonincreasing magnitude
\[
    1 = \lambda_1 \ge \lambda_2 \ge \cdots \ge \lambda_n \ge -1
\] the mixing rate is
\[
    \mu(P) = \max_{i=2,\dots,n} | \lambda_i(P) | = \max(\lambda_2, -\lambda_n)
\] See below for more information and proofs.  The smaller \( \mu(P) \)
is, the faster the Markov chain converges to its stationary
distribution.

The following is an application and motivation for this Markov chain. A
processor is located at each vertex of the graph.  Each edge represents
a direct network connection between the adjacent processors. The
processor can be a computer in a network or it can be human worker, such
as a line of assemblers in a workshop.  The line has a constant total
amount of work to be done.  Each processor has a job load or queue to
finish, say processor \( j \) has load \( q_j (t) \) at time \( t \)
where \( q_j(t) \) is a positive real number.  The load balancing is
done in steps \emph{before} any processing or work begins, so the total
amount of work to be done is constant.  At each step, the goal is shift
loads between adjacent vertices to balance the load.  In other words the
goal is to have \( q_j(t) \to (1/n) \sum_j q_{j}(0) \) as fast as
possible as \( t \to \infty \).  The results here show the balance can
be accomplished fastest by shifting one-half of the load on each vertex
to the adjacent vertices.

\subsection*{Proofs about Fastest Mixing }

\begin{lemma}
    \label{fastestmixing:lemma:one} If \( P \) is an \( n \times n \)
    symmetric stochastic matrix, then
    \[
        \mu(P) = \| P - (1/n) \vect{1}\vect{1}^T \|_2
    \] where \( \| \cdot \|_2 \) denotes the spectral norm.
\end{lemma}

\begin{remark}
    The \defn{spectral norm}%
    \index{spectral norm}%
    , also called the operator norm, is the natural norm of a matrix
    induced by the \( L^2 \) or Euclidean vector norm.  The spectral
    norm is also the maximum singular value of the matrix, that is the
    square root of the maximum eigenvalue of \( A^H A \) where \( A^H \)
    is the conjugate transpose.
\end{remark}

\begin{proof}
    \begin{enumerate}
        \item
            Note that \( \vect{1} \) is the eigenvector of \( P \)
            associated with the eigenvalue \( \lambda = 1 \) by
            equation~%
            \eqref{fastestmixing:eq:eigenvalue_one}.  Also \( (1/n)
            \vect{1}\vect{1}^T \vect{1} = \vect{1} \).
        \item
            Let \( \vect{u}^{(2)}, \dots \vect{u}^{(n)} \) be the other \(
            n-1 \) eigenvectors of \( P \) corresponding to the
            eigenvalues \( \lambda_2, \dots \lambda_n \).  Because \( P \)
            is symmetric, the eigenvectors are orthogonal.  Taking the
            inner product of \( \vect{u}^{(j)} \) with eigenvector \(
            \vect{1} \), observe that \( \sum_{\nu=1}^n u_\nu^{(j)} = 0 \)
            for \( j=2,\dots, n \).  Therefore
            \[
                (1/n) \vect{1}\vect{1}^T \vect{u}^{(j)} = \vect{0}.
            \]
        \item
            Then the eigenvalues of \( P - (1/n)\vect{1}\vect{1}^T \)
            are \( \lambda_1 = 0, \lambda_2, \dots \lambda_n \) with
            eigenvectors \( \vect{1}, \vect{u}^{(2)}, \dots, \vect{u}^{(n)}
            \) respectively.
        \item
            Since \( P - (1/n) \vect{1}\vect{1}^T \) is symmetric, it is
            equivalent to a diagonal matrix, and its spectral norm is
            equal to the maximum magnitude of its eigenvalues, i.e. \(
            \max\{ {|\lambda_2|, \dots, |\lambda_n|}\} \), which is \(
            \mu(P) \).
    \end{enumerate}
\end{proof}

\begin{lemma}
    \label{fastestmixing:lemma:two} If \( P \) is an \( n \times n \)
    symmetric stochastic matrix and if \( \vect{y} \) and \( \vect{z} \)
    in \( \mathbb{R}^n \) satisfy
    \begin{align}
        \vect{1}^T \vect{y} &= 0%
        \label{fastestmixing:eq:orthogonal}\\
        \| \vect{y} \|_2 &= 1%
        \label{fastestmixing:eq:normalize} \\
        (z_i + z_j)/2 &\le y_i y_j \text{ for \( i \), \( j \) with \( P_
        {ij} \ne 0 \) }%
        \label{fastestmixing:eq:inequality}
    \end{align}
    then \( \mu(P) \ge \vect{1}^T \vect{z} \).
\end{lemma}

\begin{proof}
    \begin{enumerate}
        \item
            Let the eigenvectors of \( P \) be \( \{ \vect{u}^{(1)} =
            \vect{1}, \vect{u}^{(2)}, \dots , \vect{u}^{(n)} \} \).  By
            the Spectral Theorem these eigenvectors are an orthonormal
            basis of \( \mathbb{R}^n \).
        \item
            Let \( \vect{y} \) be as specified in the hypotheses~%
            \eqref{fastestmixing:eq:orthogonal} and~%
            \eqref{fastestmixing:eq:normalize} and let
            \[
                \vect{y} = \sum_{\nu=1}^{n} \alpha_\nu \vect{u}^{(\nu)}.
            \]
        \item
            By hypothesis~%
            \eqref{fastestmixing:eq:normalize}, \( \| \vect{y} \|_2 =
            \sqrt{\sum_{\nu=1}^n \alpha_\nu^2 } = 1 \).
        \item
            The eigenvalues of \( P - (1/n)\vect{1}\vect{1}^T \) are \(
            \lambda_1 = 0, \lambda_2, \dots \lambda_n \) with
            eigenvectors \( \vect{1}, \vect{u}^{%
            (2)}, \dots, \vect{u}^{(n)} \) so
            \[
                (P - (1/n)\vect{1}\vect{1}^T) \left( \sum_{\nu=1}^n
                \alpha_\nu \vect {u}^ {(\nu)} \right) = \sum_{\nu=1}^n
                \alpha_\nu \lambda_\nu \vect{u}^{(\nu)} = \sum_{\nu=2}^n
                \alpha_\nu \lambda_\nu \vect{u}^ {(\nu)}.
            \] The last sum on the right side intentionally starts at \(
            2 \) because the first eigenvalue is \( 0 \).
        \item
            By Lemma~%
            \ref{fastestmixing:lemma:one}
            \[
                \mu(P) = \| P - (1/n)\vect{1}\vect{1}^T \|_2.
            \] By definition of the spectral norm
            \[
                \| P - (1/n)\vect{1}\vect{1}^T \|_2 = \max_{\|\vect{w}
                \|=1} \| (P - (1/n)\vect{1}\vect{1}^T) \vect{w}\|_2.
            \]
        \item
            \label{fastestmixing:enum:stepsix} Specializing to the
            vector \( \vect{y} \) with \( \| \vect{y} \|_2 = 1 \) and
            using the definition of the 2-norm
            \begin{align*}
                & \max_{\|\vect{w} \|=1} \| (P - (1/n)\vect{1}\vect{1}^T)
                \vect{w}\|_2 \\
                &\quad \ge \| (P - (1/n)\vect{1}\vect{1}^T) \vect{y} \|_2
                \\
                &\quad = \sqrt{\vect{y}^T (P - (1/n)\vect{1}\vect{1}^T)^T
                (P - (1/n)\vect {1}\vect{1}^T) \vect{y} } \\
                &\quad = \sqrt{\left( \sum_{\nu=1}^n \alpha_\nu \vect{u}^
                {(\nu)} \right)^T (P - (1/n)\vect{1}\vect{1}^T)^T (P - (1/n)\vect
                {1}\vect {1}^T) \left( \sum_{\nu=1}^n \alpha_\nu \vect{u}^
                {(\nu)} \right)} \\
                &\quad = \sqrt{\left( \sum_{\nu=1}^n \alpha_\nu \lambda_\nu
                \vect{u}^ {(\nu)} \right)^T \left( \sum_{\nu=1}^n \alpha_\nu
                \lambda_\nu \vect{u}^ {(\nu)} \right)} \\
                &\quad = \sqrt{\sum_{\nu=1}^n \alpha_\nu^2 \lambda_\nu^2}.
            \end{align*}
        \item
            Apply Jensen's Inequality to the concave down function \(
            \sqrt{\cdot} \) over the convex combination defined by \(
            \alpha_i^2 \) which has \( \sum_{i=1}^n \alpha_i^2 = \|
            \vect{y} \|_2 = 1 \)
            \[
                \sqrt{\sum_{i=1}^n \alpha_i^2 \lambda_i^2} \ge \sum_{i=1}^n
                \alpha_i^2 | \lambda_i | \ge \sum_{i=1}^n \alpha_i^2
                \lambda_i.
            \]
        \item
            Now unwind the expression back into vector notation, again
            using the orthogonality of the eigenvectors
            \begin{align*}
                \sum_{\nu=1}^n \alpha_\nu^2 \lambda_\nu &= \left( \sum_{i=1}^n
                \alpha_i \vect{u}^{(i)} \right)^T (P - (1/n)\vect {1}\vect
                {1}^T) \left( \sum_{i=1}^n \alpha_i \vect{u}^{(i)}
                \right) \\
                &= \vect{y}^T (P - (1/n)\vect{1}\vect{1}^T) \vect{y} \\
                &= \vect{y}^T P \vect{y} \\
                &= \sum_{i,j} P_{ij} y_i y_j.
            \end{align*}
        \item
            Now use the first inequality in step~%
            \ref{fastestmixing:enum:stepsix} and hypothesis~%
            \eqref{fastestmixing:eq:inequality}
            \begin{align*}
                \sum_{i,j} P_{ij} y_i y_j &\ge \sum_{i,j}(1/2)(z_i + z_j)P_
                {ij} \\
                &= (1/2) (\vect{z}^T P \vect{1} + \vect{1}^T P \vect{z}
                ) \\
                &= \vect{1}^T \vect{z}.
            \end{align*}
            The second equality follows from the fact that \( P \) is
            doubly-stochastic.  This establishes the lemma.
    \end{enumerate}
\end{proof}

\begin{lemma}
    \label{fastestmixing:lemma:three} The eigenvalues of the \( n \times
    n \) tridiagonal matrix
    \[
        P_{1/2} =
        \begin{pmatrix}
            1/2 & 1/2 & 0 & \cdots & 0 \\
            1/2 & 0 & 1/2 & \cdots & 0 \\
            0 & \ddots & \ddots & \ddots& 0 \\
            0 & \cdots & 1/2 & 0 & 1/2 \\
            0 & \cdots & 0 & 1/2 & 1/2
        \end{pmatrix}
    \] are \( \lambda_1 = 1 \) and \( \lambda_j = \cos\left( \frac{(j-1)
    \pi}{n} \right) \) for \( j=2, \dots, n \).
\end{lemma}

\begin{remark}
    The following proof is adapted from Feller~%
    \cite[Section XVI.4]{feller73}.  In fact, Feller finds the
    eigenvalues and eigenvectors for the more general tridiagonal matrix
    \[
        P_{p} =
        \begin{pmatrix}
            q & p & 0 & \cdots & 0 \\
            q & 0 & p & \cdots & 0 \\
            0 & \ddots & \ddots & \ddots& 0 \\
            0 & \cdots & q & 0 & p \\
            0 & \cdots & 0 & q & p
        \end{pmatrix}
        .
    \] The exercises derive the eigenvalues of this more general matrix.
\end{remark}

\begin{proof}
    \begin{enumerate}
        \item
            The proof directly finds the eigenvalues permitting solution
            of the linear system \( P_{1/2} \vect{u} = \lambda \vect{u} \).
            The problem is treated as a linear system in the \( (n-2)
            \times (n-2) \) system defined by equations \( 2, \dots n-1 \)
            in the variables \( u_2, \dots, u_{n-1} \) and then using
            the first and last equation as boundary conditions
            determining values of \( \lambda \) that permit a nontrivial
            solution.
        \item
            The equations are
            \begin{align}
                \lambda u_1 &= (1/2) u_1 + (1/2) u_2%
                \label{fastestmixing:eq:bcleft} \\
                \lambda u_j &= (1/2) u_{j-1} + (1/2) u_{j+1}, j=2,\dots.n-1%
                \label{fastestmixing:eq:middle} \\
                \lambda u_n &= (1/2) u_{n-1} + (1/2) u_n.%
                \label{fastestmixing:eq:bcright}
            \end{align}
        \item
            Equation~%
            \eqref{fastestmixing:eq:middle} is satisfied by \( u_j = s^j
            \) provided
            \[
                \lambda s = (1/2) + (1/2) s^2
            \] or \( s_{+} = \lambda + \sqrt{\lambda^2 - 1} \) and \( s_
            {-} = \lambda - \sqrt{\lambda^2 - 1} \).  Then the general
            solution is of the form \( u_j = A(\lambda) s_{+}^j + B(\lambda)
            s_{-}^j \).
        \item
            Apply the first equation~(%
            \eqref{fastestmixing:eq:bcleft}) to get
            \begin{align*}
                \lambda (A s_{+} + B s_{-} ) &= (1/2)(A s_{+} + B s_{-})
                + (1/2) (A s_{+}^2 + B s_{-}^2) \\
                0 &= A[(1-2\lambda)s_{+} + s_{+}^2] + B[(1-2\lambda)s_{-}
                + s_{-}^2] \\
                0 &= A[s_{+} - 1] + B[s_{-} -1] \\
                A[s_{+} - 1] &= -B[s_{-} -1].
            \end{align*}
        \item
            Apply the last equation~%
            \eqref{fastestmixing:eq:bcright} to get
            \begin{align*}
                \lambda (A s_{+}^{n} + B s_{-}^{n} ) &= (1/2)(A s_{+}^{n-1}
                + B s_{-}^{n-1}) + (1/2)(A s_{+}^n + B s_{-}^n) \\
                0 &= A s_{+}^{n-1} [1-2\lambda s_{+} + s_{+}] + B s_{-}^
                {n-1} [1-2\lambda s_{-} + s_{-}] \\
                0 &= A s_{+}^{n-1} [-s_{+}^2 + s_{+}] + B s_{-}^{n-1} [-s_
                {-}^2 + s_{-}] \\
                0 &= A s_{+}^{n} [-s_{+} + 1] + B s_{-}^{n} [-s_{-} + 1]
                \\
                -A s_{+}^{n} [-s_{+} + 1] &= B s_{-}^{n} [-s_{-} + 1] \\
                A [s_{+} - 1] s_{+}^{n} &= -B [s_{-} - 1] s_{-}^{n}.
            \end{align*}
        \item
            Combining these equations \( s_{+}^{n} = s_{-}^{n} \).  Note
            that \( s_{+}s_{-} =1 \), so \( s_{+}^{2n} = 1 \) and \( s_{-}^
            {2n} = 1 \).  That is, both \( s_{+} \) and \( s_{-} \) are \(
            2n \)th roots of unity.  Therefore, \( s_{+} \) and \( s_{-}
            \) can be written in the form
            \[
                \EulerE^{\I \pi j/n} = \cos\left( \frac{\pi j}{n} \right)
                + \I \sin\left( \frac{ \pi j}{n} \right)
            \] for \( j = 0, 1, 2, \dots, 2n-1 \).
        \item
            Thus, the eigenvalues \( \lambda \) must be a solution of
            \[
                s_{+}(\lambda) = \EulerE^{\I \pi j/n}
            \] or
            \begin{align*}
                \lambda + \sqrt{\lambda^2 - 1} &= \EulerE^{\I \pi j/n}
                \\
                \sqrt{\lambda^2 - 1} &= -\lambda + \EulerE^{\I \pi j/n}
                \\
                \lambda^2 - 1 &= \lambda^2 - 2\lambda \EulerE^{\I \pi
                j/n} + \EulerE^{2 \I \pi j/n} \\
                -1 &= - 2\lambda \EulerE^{\I \pi j/n} + \EulerE^{2 \I
                \pi j/n} \\
                \frac{1}{2 \EulerE^{\I \pi j/n} } + \frac{\EulerE^{\I
                \pi j/n}}{2} &= \lambda \\
                \frac{\EulerE^{-\I \pi j/n} }{2} + \frac{\EulerE^{\I \pi
                j/n}}{2} &= \lambda \\
                \cos{\pi j/n } &= \lambda.  \\
            \end{align*}
        \item
            So to each \( j \) there is a root \( \lambda_j \), namely
            \[
                \lambda_j = \cos( \pi j/n ) \qquad j = 0,1,2, \dots,
                2n-1.
            \] However, some of these roots are repeated, since
            \[
                \cos( \pi j/n ) = \cos( \pi (2n-j)/n )
            \] for \( j = 1, \dots, n \).  The \( n \) eigenvalues are \(
            \lambda_1 = 1 \) and \( \lambda_j = \cos\left( \frac{(j-1)
            \pi}{n} \right) \) for \( j=2, \dots, n \).
    \end{enumerate}
\end{proof}

\begin{remark}
    The following lemma is a purely technical lemma about sums of
    cosines of arithmetic sequences.  The next two lemmas use results of
    this lemma to shorten the proofs.
\end{remark}

\begin{lemma}
    \label{fastestmixing:lemma:cossum}
    \begin{enumerate}
        \item
            For an integer \( j \)
            \[
                \sum_{\ell=1}^{n} \cos((2 \ell-1)(j-1)\pi/n) = 0.
            \]
        \item
            In the special case \( j=2 \)
            \[
                \sum_{\ell=1}^{n} \cos((2 \ell-1)\pi/n) = 0.
            \]
        \item
            For an integer \( j \)
            \[
                \sum_{\ell=1}^{n} \cos^2((2 \ell-1)(j-1)\pi/(2n)) =
                \frac{n}{2}.
            \]
    \end{enumerate}
\end{lemma}

\begin{proof}
    \begin{enumerate}
        \item
            \begin{align*}
                &\sum_{\ell=1}^{n} \cos((2\ell-1)(j-1)\pi/n) \\
                &\quad = \Re \left ( \sum_{\ell=1}^n \exp (\I (2\ell-1)(j-1)\pi/n)
                \right) \\
                &\quad = \Re \left( \sum_{\ell=1}^n \exp(2\ell(j-1)\pi\I
                /n) \exp(-\I (j-1)\pi/n) \right) \\
                &\quad = \Re \left( \exp(-\I(j-1)\pi/n) \sum_{\ell=1}^n
                \left[ \exp(2 (j-1)\pi \I/n) \right]^\ell \right) \\
                &\quad = \Re \left( \exp(-\I(j-1)\pi/n) \exp( 2 (j-1)\pi
                \I/n) \sum_{\ell=1}^{n} \left[ \exp( 2 (j-1)\pi \I/n)
                \right]^{\ell-1} \right) \\
                &\quad = \Re \left( \exp(-(j-1)\pi\I/n) \exp(2 (j-1)\pi
                \I/n) \sum_{\ell=0}^{n-1} \left[ \exp( 2 (j-1)\pi \I/n)
                \right]^{\ell} \right) \\
                &\quad = \Re \left( \exp(\I (j-1)\pi/n) \frac{1 - \left[
                \exp( 2 (j-1)\pi \I /n) \right]^n}{1 - \exp ( 2(j-1)\pi
                \I/n)} \right) \\
                &\quad = \Re \left( \exp(\I (j-1)\pi/n) \frac{1 - \exp(2
                (j-1)\pi \I)}{1 - \exp(2 (j-1)\pi \I/n)} \right) \\
                &\quad = 0.
            \end{align*}
        \item
            This is an immediate corollary of the previous with \( j=2 \).
        \item
            Use the half-angle formula for the cosine function so
            \[
                \cos^2((2 \ell-1)(j-1)\pi/(2n)) = \frac{1}{2} + \cos((2
                \ell-1)(j-1)\pi/n)
            \] and the result follows from the first part.
    \end{enumerate}

\end{proof}
\begin{lemma}
    \label{fastestmixing:lemma:threepointfive} The normalized
    eigenvectors of the \( n \times n \) tridiagonal matrix
    \[
        P_{1/2} =
        \begin{pmatrix}
            1/2 & 1/2 & 0 & \cdots & 0 \\
            1/2 & 0 & 1/2 & \cdots & 0 \\
            0 & \ddots & \ddots & \ddots& 0 \\
            0 & \cdots & 1/2 & 0 & 1/2 \\
            0 & \cdots & 0 & 1/2 & 1/2
        \end{pmatrix}
    \] are
    \begin{align*}
        \lambda_1 &= 1, & \vect{u}^{(1)} &= \frac{1}{\sqrt{n}}\vect{1}
        \\
        \lambda_j &= \cos\left( \frac{(j-1) \pi}{n} \right), & u^{(j)}_k
        &= \sqrt{\frac{2}{n}} \cos\left( \frac{(2k-1)(j-1)\pi}{2n}
        \right)
    \end{align*}
    for \( j=2, \dots, n \) and \( k=1,\dots, n \).
\end{lemma}

\begin{proof}
    \begin{enumerate}
        \item
            Since \( P_{1/2} \) is symmetric, already it is established
            that for \( \lambda_1 = 1 \), the corresponding normalized
            eigenvector is \( \vect{u}^{(1)} = \frac{1}{\sqrt{n}}\vect{1}
            \).
        \item
            For \( j \ge 2 \), \( \lambda_j = \cos\left( \frac{(j-1) \pi}
            {n} \right) \).  The proof of Lemma~%
            \ref{fastestmixing:lemma:three} shows the corresponding
            unnormalized eigenvectors have components
            \[
                u^{(j)}_k = A(\lambda_j) s_{+}^k(\lambda_j) + B(\lambda_j)s_
                {-}^k(\lambda_j)
            \] where \( s_{\pm}(\lambda_j) = \lambda_j \pm \sqrt{\lambda_j^2
            - 1} = \cos((j-1) \pi/n) \pm \I \sin((j-1) \pi/n) = \EulerE^
            {\pm (j-1) \pi \I} \).
        \item
            The coefficients \( A(\lambda_j) \) and \( B(\lambda_j) \)
            are determined from the boundary conditions~%
            \eqref{fastestmixing:eq:bcleft} and~%
            \eqref{fastestmixing:eq:bcright}.  But since
            \[
                u^{(j)}_k = A(\lambda_j) \left[ s_{+}^k(\lambda_j) +
                \frac{B(\lambda_j)}{A(\lambda_j)}s_{-}^k(\lambda_j)
                \right]
            \] and any scalar multiple of an eigenvector is an
            eigenvector, only the ratio \( \frac{B(\lambda_j)}{A(\lambda_j)}
            \) is needed.
        \item
            For notational simplicity, temporarily drop the dependence
            of \( s_{\pm} \) on \( \lambda_j \).  Expanding~\eqref{fastestmixing:eq:bcleft}
            \[
                2\lambda A s_+ + 2 \lambda B s_{-} = A s_+ + B s_- + A s_+^2
                + B s_-^2.
            \]
        \item
            Rearranging and using \( s_{\pm}^2 - 2\lambda s_{\pm} = -1 \)
            \[
                \frac{B}{A} = - \frac{s_+ - 1}{s_- - 1}
            \] and the components of the unnormalized eigenvector are
            \[
                u_k^{(j)} = \left[ s_+^k(\lambda_j) - \frac{s_+(\lambda_j)
                - 1}{s_-(\lambda_j) - 1} s_-^k(\lambda_j) \right].
            \] Factoring out \( s_+ - 1 \) and remembering that any
            scalar multiple of an eigenvector is an eigenvector, the
            components of the unnormalized eigenvector are
            \[
                u_k^{(j)} = \left[ \frac{s_+^k(\lambda_j)}{s_+(\lambda_j)
                - 1} - \frac{s_-^k(\lambda_j)}{s_-(\lambda_j) - 1}
                \right].
            \]
        \item
            Recalling \( s_{\pm}(\lambda_j) = \EulerE^{\pm \I (j-1)\pi/n}
            \), each fraction can be simplified by multiplying numerator
            and denominator by the complex conjugate of the denominator
            \[
                \frac{\EulerE^{\pm \I (j-1) \pi k/n}}{\EulerE^{\pm \I (j-1)
                \pi/n} - 1} \cdot \frac{\EulerE^{\mp \I (j-1) \pi k/n}-1}
                {\EulerE^{\mp \I (j-1) \pi/n} - 1} = \frac{\EulerE^{\pm
                \I \pi (j-1) (k-1)/n} - \EulerE^{\pm \I \pi (j-1) k/n}}{2-
                2\cos(\pi/n)}
            \]
        \item
            Once again remembering that any scalar multiple of an
            eigenvector is an eigenvector, the constant \( 2- 2\cos(\pi/n)
            \) in the denominator can be ignored.
        \item
            This leaves
            \begin{align*}
                u_k^{(j)} &= (\EulerE^{+ \I (j-1)(k-1)\pi/n} - \EulerE^{-
                \I \pi (j-1)(k-1)/n}) - (\EulerE^{+ \I (j-1) k \pi /n} -
                \EulerE^{- \I (j-1) k \pi /n}) \\
                &= 2\sin( (j-1) (k-1) \pi /n) - 2\sin( (j-1) k\pi /n) \\
                &= 4 \sin( (j-1) (-1) \pi/(2n)) \cos((j-1) (2k-1) \pi /(2n)
                ).
            \end{align*}
        \item
            Finally, factoring out the constant \( -4 \sin( (j-1)\pi/(2n))
            \), the \( k \)th component of the eigenvector corresponding
            to \( \lambda_j \) is
            \[
                u^{(j)}_k = \cos\left( \frac{(2k-1)(j-1)\pi}{2n} \right).
            \]
        \item
            Using Lemma~%
            \ref{fastestmixing:lemma:cossum}, the norm is \( \sqrt{n/2} \)
            and the \( k \)th component of the normalized eigenvector
            corresponding to \( \lambda_j \) is
            \[
                u^{(j)}_k = \sqrt{\frac{2}{n}} \cos\left( \frac{(2k-1)(j-1)\pi}
                {2n} \right).
            \]
    \end{enumerate}
\end{proof}

\begin{theorem}
    \label{fastestmixing:theorem:main} The value \( \mu(P_{1/2}) = \cos(\pi/n)
    \) for
    \[
        P_{1/2} =
        \begin{pmatrix}
            1/2 & 1/2 & & & \\
            1/2 & 0 & 0 & & \\
            & \ddots & \ddots & \ddots& \\
            & & 1/2 & 0 & 1/2 \\
            & & & 1/2 & 1/2
        \end{pmatrix}
    \] is the smallest among all symmetric stochastic tridiagonal
    matrices.
\end{theorem}

See Figure~%
\ref{fastestmixing:fig:fastest_mixing_graph} for the graph associated
with this transition probability matrix.
\begin{figure}
\begin{asy}
settings.outformat = "pdf";

size(5inches);

real myfontsize = 12;
real mylineskip = 1.2*myfontsize;
pen mypen = fontsize(myfontsize, mylineskip);
defaultpen(mypen);

real marge= 1mm;
pair z1 = (1,0), z2 = (3,0), z3 = (5,0);
pair znm2 = (8,0), znm1 = (11,0), zn = (14,0);
transform r=scale(1.0);

object state1 = draw("$1$", roundbox, z1, marge),
       state2 = draw("$2$", roundbox, z2, marge),
       state3 = draw("$3$", roundbox, z3, marge),
statenm2 = draw("$\scriptstyle{n-2}$", roundbox, znm2, marge),
       statenm1 = draw("$\scriptstyle{n-1}$", roundbox, znm1, marge),
       staten = draw("$n$", roundbox, zn, marge);

add(new void(picture pic, transform t) {
    draw(pic, r*Label("$1/2$"), point(state1, E, t)..point(state2, W, t), Arrows);
    draw(pic, r*Label("$1/2$"), point(state2, E, t)..point(state3, W, t), Arrows);

    draw(pic, r*Label(""), point(state3, E, t)..point(statenm2, W, t), dashed, Arrows);

    draw(pic, Label("$1/2$"), point(statenm2, E, t)..point(statenm1, W, t), Arrows);
    draw(pic, Label("$1/2$"), point(statenm1, E, t)..point(staten, W, t), Arrows);
 });

add(new void(picture pic, transform t) {
    draw(pic, r*Label("$1/2$"), point(state1, NE, t){NE}..point(state1, NW, t), Arrow);
    draw(pic, Label("$1/2$"), point(staten,   NE, t){NE}..point(staten,   NW, t), Arrow);
 });

label("$1/2$", (z1+z2)/2, 1.5N);
label("$1/2$", (z3+z2)/2, 1.5N);

label("$1/2$", (znm2+znm1)/2, 1.5N);
label("$1/2$", (zn+znm1)/2, 1.5N+0.1E);
\end{asy}
    \caption{The random walk graph with fastest mixing.%
    \label{fastestmixing:fig:fastest_mixing_graph}}
\end{figure}

\begin{proof}
    \begin{enumerate}
        \item
            The proof proceeds by constructing a pair of vectors \(
            \vect{y} \) and \( \vect{z} \) that satisfy the assumptions
            of Lemma~%
            \ref{fastestmixing:lemma:two} for any symmetric tridiagonal
            stochastic matrix \( P \).  Furthermore, \( \vect{1}^T \vect
            {z} = \cos(\pi/n) \), so the mixing rate \( \mu(P_{1/2}) =
            \cos(\pi/n) \) is the fastest possible.
        \item
            By Lemma~%
            \ref{fastestmixing:lemma:three} and the definition of the
            mixing rate
            \[
                \mu(P_{1/2}) = \lambda_2 = -\lambda_n = \cos(\pi/n).
            \] Take \( \vect{y} = \vect{u}^{(2)} \), the normalized
            eigenvector of \( P_{1/2} \) corresponding to \( \lambda_2 \),
            so the assumptions~%
            \eqref{fastestmixing:eq:orthogonal} and~%
            \eqref{fastestmixing:eq:normalize} in Lemma~%
            \ref{fastestmixing:lemma:two} are automatically satisfied.
        \item
            Take \( \vect{z} \) to be the vector with
            \[
                z_j = \frac{1}{n} \left[ \cos\left(\frac{\pi}{n}\right)
                + \cos\left( \frac{(2j-1)\pi }{n } \right) \bigg/ \cos\left
                ( \pi/n \right) \right]
            \] for \( j = 1, \dots, n \).
        \item
            By Lemma~%
            \ref{fastestmixing:lemma:cossum}
            \[
                \sum_{j=1}^{n} \cos((2j-1)\pi/n) = 0.
            \] Then \( \vect{1}^T \vect{z} = \cos(\pi/n) \).
        \item
            Now check that \( \vect{y} \) and \( \vect{z} \) satisfy
            hypothesis~%
            \eqref{fastestmixing:eq:inequality} of Lemma~%
            \ref{fastestmixing:lemma:two}.
            \begin{multline*}
                \frac{z_j + z_{j+1}}{2} = \\
                \frac{1}{n} \left[ \cos\left(\frac {\pi} {n}\right) +
                \frac{1}{2} \left( \cos\left( \frac{(2j-1)\pi }{n }
                \right) + \cos\left( \frac{(2j+1)\pi }{n } \right)
                \right) \bigg/ \cos\left(\pi/n \right) \right].
            \end{multline*}
        \item
            Using the cosine sum formula
            \[
                \frac{1}{2} \left( \cos\left( \frac{(2j-1)\pi }{n }
                \right) + \cos\left( \frac{(2j+1)\pi }{n } \right)
                \right) = \cos\left( \frac{\pi}{n} \right) \cos\left(
                \frac{2 j \pi}{n} \right)
            \] this simplifies to
            \[
                \frac{z_j + z_{j+1}}{2} = \frac{1}{n} \left[ \cos\left(
                \frac{\pi} {n} \right) + \cos\left( \frac{2j\pi}{n}
                \right) \right].
            \]
        \item
            Using the cosine sum formula again
            \begin{multline*}
                \frac{1}{n} \left[ \cos\left( \frac{\pi}{n} \right) +
                \cos\left( \frac{2j\pi}{n} \right) \right] = \frac{2}{n}
                \cos\left( \frac{(2j-1) \pi }{2n} \right) \cos\left(
                \frac{(2j+1) \pi }{2n} \right) \\
                = y_{j} y_{j+1}.
            \end{multline*}
        \item
            Therefore \emph{equality} in inequality~%
            \eqref{fastestmixing:eq:inequality} holds for the adjacent
            entries resulting from the nonzero subdiagonal and
            superdiagonal entries.  For the diagonal entries, check that
            \( (z_j + z_j)/2 = z_j \le y_j^2 \).  That is, check that
            \[
                \cos\left( \frac{\pi}{n} \right) + \cos\left( \frac{(2j-1)\pi}
                {2n} \right) \bigg/ \cos\left( \frac{\pi}{2n} \right)
                \le 2 \cos^2\left( \frac{(2j-1)\pi}{n} \right).
            \]
        \item
            Using the double-angle formula for the cosine
            \[
                2 \cos^2\left( \frac{(2j-1)\pi}{n} \right) = 1 + \cos\left
                ( \frac{(2j-1)\pi}{2n} \right).
            \] Therefore, the following inequality needs to be checked
            \[
                \cos\left( \frac{\pi}{n} \right) + \cos\left( \frac{(2j-1)\pi}
                {n} \right) \bigg/ \cos\left( \frac{\pi}{n} \right) \le
                1 + \cos\left( \frac{(2j-1)\pi}{n} \right).
            \]
        \item
            Moving all terms to the right the inequality to check is
            \[
                1 - \cos\left( \frac{\pi}{n} \right) - \cos\left( \frac{%
                (2j-1)\pi} {n} \right) \bigg/ \cos\left( \frac{\pi}{n}
                \right)+ \cos\left( \frac{(2j-1)\pi}{n} \right) \ge 0.
            \]
        \item
            The left side can be factored as
            \[
                \left[ 1 - \cos\left( \frac{\pi}{n} \right) \right]
                \left[ 1 - \cos\left( \frac{(2j-1)\pi}{n} \right) \bigg/
                \cos\left( \frac{\pi} {n} \right) \right].
            \]
        \item
            This is true because
            \[
                \cos\left(\frac{(2j-1)\pi}{n} \right) \bigg/ \cos\left(
                \frac{\pi} {n} \right) \le 1
            \] for \( j = 1, \dots, n \).
    \end{enumerate}
\end{proof}

\hr

\visual{Section Starter Question}{../../../../CommonInformation/Lessons/question_mark.png}
\section*{Section Ending Answer}

Since the transition probability matrix \( P \) is stochastic, \( P
\vect{1} = \vect{1} \).  Since the matrix \( P \) is symmetric, \( \vect
{1}^{T} P = \vect{1}^{T} \).  Then normalizing, \( (1/n)\vect{1}^{T} P =
(1/n)\vect{1}^{T} \) so the uniform distribution is a stationary
distribution.  Intuitively, this make sense since if the matrix is
symmetric, the transition probabilities are the same to go ``left'' as
to go ``right'', so the stationary distribution will ``even out'' to be
uniform.  The rate of convergence is determined by the eigenvalues, and
the eigenvalue with second largest magnitude (since the largest
eigenvalue is \( 1 \)) will dominate the asymptotic rate.

\subsection*{Sources}

This section is adapted from the article by Boyd, Diaconis, Sun and
Xiao
\cite{doi:10.1080/00029890.2006.11920281}.

\hr

\visual{Algorithms, Scripts, Simulations}{../../../../CommonInformation/Lessons/computer.png}
\section*{Algorithms, Scripts, Simulations}

\subsection*{Algorithm}

\subsection*{Scripts}

%% \input{ _scripts}

\hr

\visual{Problems to Work}{../../../../CommonInformation/Lessons/solveproblems.png}
\section*{Problems to Work for Understanding}

\renewcommand{\theexerciseseries}{}
\renewcommand{\theexercise}{\arabic{exercise}}

\begin{exercise}
    Show that balancing the workload of a line of processors by shifting
    one-half of the load on each vertex from the more loaded to the less
    loaded processor can be represented by the tridiagonal matrix \( P_{1/2}
    \).
\end{exercise}
\begin{solution}
    Consider a representative node \( j \) with \( 1 \le j \le n-1 \)
    with load \( q_j \) and neighboring node with load \( q_{j+1} \).
    After one Markov chain step by \( P_{1/2} \) of load exchange the
    expected gain would be \( \frac{1} {2} q_{j+1} - \frac{1}{2} q_{j} \).
    If \( q_{j+1} > q_j \), the gain is positive, one-half of the load
    imbalance shifts from \( j+1 \) to \( j \).  If \( q_{j+1} < q_j \),
    the gain is negative, one-half of the load imbalance shifts from \(
    j \) to \( j+1 \).  For the end node \( n \), the calculation is
    similar.
\end{solution}

\begin{exercise}
    Show that \( (1/n) \vect{1}\vect{1}^T \vect{1} = \vect{1} \).
\end{exercise}
\begin{solution}
    The matrix \( \vect{1}\vect{1}^T \) has \( 1 \) in every entry. The
    matrix-vector product \( \vect{1}\vect{1}^T \vect{1} \) sums the
    rows, giving \( n \vect{1} \), then \( (1/n) \vect{1}\vect{1}^T
    \vect{1} = \vect{1} \).
\end{solution}

\begin{exercise}
    Devise the exchange rules for an urn model with the \( n \times n \)
    transition probability matrix \( P_{p} \).
\end{exercise}
\begin{solution}
    There are two urns I and II containing a total of \( n \) balls. The
    state of the Markov chain is the number of balls in urn I so the
    number of states is \( n+1 \).  At each turn, with probability \( p \)
    a ball is taken from urn II and moved to urn I or with complementary
    probability \( q \) a ball is moved from urn I to urn II\@.  If
    there are no balls in urn I then with probability \( q \) urn I
    stays empty.  Similarly for the last state if urn I has all \( n \)
    balls.
\end{solution}
\begin{exercise}
    Show that
    \[
        \frac{1}{2} \left( \cos\left( \frac{(2j-1)\pi }{n } \right) +
        \cos\left( \frac{(2j+1)\pi }{n } \right) \right) = \cos\left(
        \frac{\pi}{n} \right) \cos\left( \frac{2 j \pi}{n} \right)
    \] and
    \[
        \frac{1}{n} \left[ \cos\left( \frac{\pi}{n} \right) + \cos\left(
        \frac{2j\pi}{n} \right) \right] = \frac{2}{n} \cos\left( \frac{(2j
        - 1) \pi }{2n} \right) \cos\left( \frac{(2j+1) \pi }{2n} \right).
    \]
\end{exercise}
\begin{solution}
    Use the sum-to-product identity for cosines:
    \[
        \cos( \theta) + \cos(\phi) = 2 \cos \left( \frac{\theta + \phi}{2}
        \right) \cdot \cos \left( \frac{\theta - \phi}{2} \right)
    \] to directly get
    \begin{align*}
        &\frac{1}{2} \left( \cos\left( \frac{(2j-1)\pi }{n } \right) +
        \cos\left( \frac{(2j+1)\pi }{n } \right) \right) \\
        &\qquad = \cos\left( \frac{1}{2} \left[ \frac{(2j-1)\pi}{n} +
        \frac{(2j+1)\pi}{n}\right] \right) \cos\left( \frac{1}{2} \left[
        \frac{(2j-1)\pi}{n} - \frac{(2j+1)\pi}{n}\right]\right)\\
        &\qquad = \cos\left( \frac{\pi}{n} \right) \cos\left( \frac{2 j
        \pi}{n} \right)
    \end{align*}
    and
    \begin{align*}
        &\frac{1}{n} \left[ \cos\left( \frac{\pi}{n} \right) + \cos\left
        ( \frac{2j\pi}{n} \right) \right] \\
        &\qquad = \frac{2}{n} \cos\left( \frac{1}{2} \left[ \frac{2j\pi}
        {n} + \frac{\pi}{n}\right] \right) \cos\left( \frac{1}{2} \left[
        \frac{2j\pi}{n} - \frac{\pi}{n}\right]\right) \\
        &\qquad = \frac{2}{n} \cos\left( \frac{(2j - 1) \pi }{2n} \right)
        \cos\left( \frac{(2j+1) \pi }{2n} \right).
    \end{align*}
\end{solution}

\begin{exercise}
    Show that the four eigenvalues of
    \[
        P_{1/2} =
        \begin{pmatrix}
            1/2 & 1/2 & 0 & 0 \\
            1/2 & 0 & 1/2 & 0 \\
            0 & 1/2 & 0 & 1/2 \\
            0 & 0 & 1/2 & 1/2
        \end{pmatrix}
    \] are \( \cos\left( \frac{(j-1) \pi}{n} \right) \) for \( j=1,
    \dots, 4 \).
\end{exercise}
\begin{solution}
    Directly computing, the characteristic polynomial is \( p(x) = \frac
    {1}{2} x(x-1)(2x^2 - 1) \).  The eigenvalues are \( 1, \sqrt{2}/2, 0
    -\sqrt{2}/2 \) corresponding to \( \cos(0 \cdot \pi/4) \), \( \cos(1
    \cdot \pi/4) \), \( \cos(2 \cdot \pi/4) \), \( \cos(3 \cdot \pi/4) \)
    respectively.
\end{solution}

\begin{exercise}
    Show that the eigenvalues and for the more general tridiagonal
    matrix
    \[
        P_p =
        \begin{pmatrix}
            q & p & 0 & \cdots & 0 \\
            q & 0 & p & \cdots & 0 \\
            0 & \ddots & \ddots & \ddots& 0 \\
            0 & \cdots & q & 0 & p \\
            0 & \cdots & 0 & q & p
        \end{pmatrix}
    \] are \( \lambda_1 = 1 \) and \( \lambda_j = 2 \sqrt{pq} \cos\left(
    \frac{(j-1) \pi}{n} \right) \) for \( j=2, \dots, n \).  Then show
    directly that \( P_{1/2} \) with \( p = q = \frac{1}{2} \) is the
    fastest mixing among all such tridiagonal matrices \( P \).
\end{exercise}

\begin{solution}
    Consider the \( n \times n \) tridiagonal matrix
    \[
        \begin{pmatrix}
            q & p & 0 & \cdots & 0 \\
            q & 0 & p & \cdots & 0 \\
            0 & \ddots & \ddots & \ddots& 0 \\
            0 & \cdots & q & 0 & p \\
            0 & \cdots & 0 & q & p
        \end{pmatrix}
        .
    \]

    The equations connecting the eigenvalue \( \lambda \) and the
    components \( u_i \) of the eigenvectors are
    \begin{align}
        \lambda u_1 &= q u_1 + p u_2%
        \label{fastestmixing:eq:bcone} \\
        \lambda u_2 &= q u_{1} + p u_{3}%
        \label{fastestmixing:eq:ordertwo}\\
        \vdots &\quad \vdots \\
        \lambda u_{n-1} &= q u_{n-2} + p u_{n}%
        \label{fastestmixing:eq:orderthree} \\
        \lambda u_n &= q u_{n-1} + p u_n.%
        \label{fastestmixing:eq:bcfour}
    \end{align}

    Equations~\eqref{fastestmixing:eq:ordertwo} through~\eqref{fastestmixing:eq:orderthree}
    are satisfied by \( u_j = s^j \) provided
    \begin{align*}
        \lambda s^2 &= q s + p s^3 \\
        \vdots &\quad \vdots \\
        \lambda s^{n-1} &= q s^{n-2} + p s^n.  \\
    \end{align*}
    These are all equivalent to \( \lambda s = q + p s^2 \), or \( p s^2
    - \lambda s + q = 0 \), so the solutions are \( s_{+} = \frac{\lambda
    + \sqrt{\lambda^2 - 4pq}}{2p} \) and \( s_{-} = \frac{\lambda -
    \sqrt{\lambda^2 - 4pq}}{2p} \).  Then the solution is of the form
    \begin{align*}
        u_1 &= A(\lambda) s_{+} + B(\lambda) s_{-} \\
        u_2 &= A(\lambda) s_{+}^2 + B(\lambda) s_{-}^2 \\
        \vdots &\quad \vdots \\
        u_{n-1} &= A(\lambda) s_{+}^{n-1} + B(\lambda) s_{-}^{n-1} \\
        u_n &= A(\lambda) s_{+}^n + B(\lambda) s_{-}^n.  \\
    \end{align*}
    Apply equation~\eqref{fastestmixing:eq:bcone} to get
    \begin{align}
        \lambda (A s_{+} + B s_{-} ) &= q(A s_{+} + B s_{-}) + p(A s_{+}^2
        + B s_{-}^2 ) \notag \\
        0 &= A[(\lambda-q)s_{+} - p s_{+}^2] + B[(\lambda-q)s_{-} - p s_
        {-}^2] \notag \\
        0 &= A[q(s_{+} - 1)] + B[q(s_{-} -1)].%
        \label{fastestmixing:eq:leftrelation}
    \end{align}
    Applying the last equation~\eqref{fastestmixing:eq:bcfour} to get
    \begin{align}
        \lambda (A s_{+}^{n} + B s_{-}^{n} ) &= q(A s_{+}^{n-1} + B s_{-}^
        {n-1}) + p(A s_{+}^n + B s_{-}^n) \notag \\
        0 &= A s_{+}^{n-1} [(\lambda s_{+} - q) - p s_{+}] + B s_{-}^{n-1}
        [(\lambda s_{-}- q) - p s_{-}] \notag \\
        0 &= A s_{+}^{n-1} [p s_{+}^2 - p s_{+}] + B s_{-}^{n-1} [p s_{-}^2
        - p s_{-}] \notag \\
        0 &= A s_{+}^n p [s_{+} - 1] + B s_{-}^n p [s_{-} - 1]%
        \label{fastestmixing:eq:bcrelation} \\
        - A s_{+}^n p [ s_{+} - 1] &= B s_{-}^n p[ s_{-} - 1].
    \end{align}

    Substituting equation~\eqref{fastestmixing:eq:leftrelation} get
    \[
        s_{+}^n = s_{-}^n.
    \] Note that \( s_{+}s_{-} =q/p \) so \( s_{+}^{2n} = (q/p)^n \) and
    \( s_{-}^{2n} = (q/p)^n \).  That is, both \( s_{+} \) and \( s_{-} \)
    are \( (2n) \)th roots of \( (q/p)^n \).  Therefore, \( s_{+} \) and
    \( s_{-} \) can be written in the form
    \[
        \EulerE^{\I \pi j/n} = \sqrt{q/p} \left( \cos\left( \frac{ \pi j}
        {n} \right) + \I \sin\left( \frac{\pi j}{n} \right) \right)
    \] for \( j = 0, 1, 2, \dots, 2n-1 \).

    Thus, the eigenvalues \( \lambda \) are among the solutions of
    \[
        s_{+}(\lambda) = \sqrt{q/p} \EulerE^{\I \pi j/n}
    \] or
    \begin{align*}
        \frac{\lambda + \sqrt{\lambda^2 - 4pq}}{2p} &= \sqrt{\frac{q}{p}}\EulerE^
        {\I \pi j/n} \\
        \lambda + \sqrt{\lambda^2 - 4pq} &= \sqrt{4pq}\EulerE^{\I \pi
        j/n} \\
        \sqrt{\lambda^2 - 4pq} &= -\lambda + \EulerE^{\I \pi j/n} \\
        \lambda^2 - 4pq &= \lambda^2 - 2\lambda \sqrt{4pq} \EulerE^{\I
        \pi j/n} + 4pq \EulerE^{2 i \pi j/n} \\
        -4pq &= - 2\lambda \sqrt{4pq} \EulerE^{\I \pi j/n} + 4pq \EulerE^
        {2 i \pi j/n} \\
        \frac{4pq}{ 2 \sqrt{4pq} \EulerE^{\I \pi j/n} } + \frac{4pq
        \EulerE^{\I \pi j/n}}{2 \sqrt{4pq}} &= \lambda \\
        \sqrt{4pq} \left( \frac{\EulerE^{-i \pi j/n} }{2} + \frac{\EulerE^
        {\I \pi j/n}}{2} \right) &= \lambda \\
        \sqrt{4pq} \cos{\pi j/n } &= \lambda.  \\
    \end{align*}
    So to each \( j \) there is a root \( \lambda_j \), namely
    \[
        \lambda_j = \sqrt{4pq} \cos( \pi j/n ) \qquad j = 0,1,2, \dots,
        2n-1.
    \] However, some of these roots are repeated, since
    \[
        \cos( \pi j/n ) = \cos( \pi (2n-j)/n )
    \] for \( j = 1, \dots, n \).  So there are \( n \) eigenvalues \(
    \lambda_1 = 1 \) and \( \lambda_j = \sqrt{4pq} \cos\left( \frac{(j-1)
    \pi}{n} \right) \) for \( j=2, \dots, n \).  This is maximized for \(
    p = q = 1/2 \).
\end{solution}

\begin{exercise}
    Use mathematical software to numerically evaluate the eigenvalues of
    the matrices \( P_{1/2} \) for sizes \( n = 2, \dots 8 \) and show
    that the values agree with the exact eigenvalues in Lemma~%
    \ref{fastestmixing:lemma:three}.
\end{exercise}
\begin{solution}
    A simple R program using only base functions:
\begin{lstlisting}
r <- 0
p <- 1/2
q <- 1 - p

for (n in 2:8) {

   P <- diag(r, n, n)
   P[ row(P) - col(P)  == 1] <- p
   P[ row(P) - col(P)  == -1] <- q
   P[1, 1] = q
   P[n, n] = p

   cat("Size of matrix: ", n , "Eigenvalues: ",
        unlist(eigen( P, only.values=TRUE )), "\n")
}
\end{lstlisting}
\end{solution}

\hr

\visual{Books}{../../../../CommonInformation/Lessons/books.png}
\section*{Reading Suggestion:}

\bibliography{../../../../CommonInformation/bibliography}

%   \begin{enumerate}
%     \item
%     \item
%     \item
%   \end{enumerate}

\hr

\visual{Links}{../../../../CommonInformation/Lessons/chainlink.png}
\section*{Outside Readings and Links:}
\begin{enumerate}
    \item
    \item
    \item
    \item
\end{enumerate}

\section*{\solutionsname} \loadSolutions

\hr

\mydisclaim \myfooter

Last modified:  \flastmod

\end{document}

